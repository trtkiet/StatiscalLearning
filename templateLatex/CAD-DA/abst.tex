
We propose a novel statistical method for testing the results of anomaly detection (AD) under domain adaptation (DA), which we call CAD-DA---controllable AD under DA. The distinct advantage of the CAD-DA lies in its ability to control the probability of misidentifying anomalies under a pre-specified level $\alpha$ (e.g., 0.05). The challenge within this DA setting is the necessity to account for the influence of DA to ensure the validity of the inference results. We overcome the challenge by leveraging the concept of Selective Inference to handle the impact of DA. To our knowledge, this is the first work capable of conducting a valid statistical inference within the context of DA. We evaluate the performance of the CAD-DA method on both synthetic and real-world datasets.

%Detecting anomalies (AD) through domain adaptation (DA) from source data to target data is a critical task in machine learning, especially in scenarios where the target data is limited. However, existing methods lack the capability to reliably guarantee the effectiveness of AD under DA. In this paper, we introduce a novel statistical method to control AD reliability under DA, named CAD-DA (controllable AD-DA). The key strength of CAD-DA lies in its ability to maintain the probability of misidentifying anomalies below a pre-specified level while maximizing the true detection rate. In comparison to existing literature on controllable AD, CAD-DA presents a unique challenge in addressing the effect of DA to ensure the validity of the guarantee. We overcome this challenge by leveraging the Selective Inference (SI) framework to handle the influence of DA. Specifically, we carefully examine the selection strategy for Optimal Transport-driven DA methods, whose operations can be characterized by linear/quadratic inequalities. Our results demonstrate that achieving controllable AD-DA is indeed possible. Furthermore, we enhance the true detection rate under SI by implementing a more strategic selection strategy.
%Experiments conducted on both synthetic and real-world datasets robustly support our theoretical results, showcasing that our proposed method achieves the best CAD-DA performance. To the best of our knowledge, this is the first work capable of conducting valid statistical inference within the context of DA.


% \htcomment{
% begin with something like this to set the stage

% AD under DA is important, but unlike usual AD, there are no existing methods for \textit{controllable} AD under DA. 

% \red{Duy: updated}
% }

%\vspace{-8pt}
%
%\red{Anomaly detection (AD) under domain adaptation (DA) is a critical task in machine learning, %from source data to target data is a critical task in machine learning
%especially when dealing with limited target data.
%%
%However, existing methods lack the capability to guarantee the reliability of AD under DA. 
%%
%In this paper, we introduce a novel statistical method to control AD reliability under DA, named CAD-DA (controllable AD-DA). 
%%
%The key strength of CAD-DA lies in its ability to control the probability of misidentifying anomalies below a pre-specified level while maximizing the true detection rate. 
%%
%Compared to the literature on controllable AD, CAD-DA presents a unique challenge in addressing the effect of DA to ensure the validity of the guarantee. 
%%
%We overcome this challenge by leveraging the Selective Inference (SI) framework to hand le the influence of DA. 
%%
%Specifically, we carefully examine the selection strategy for Optimal Transport-based DA, whose operations can be characterized by linear/quadratic inequalities. 
%%
%We prove that achieving controllable AD-DA is indeed possible. 
%%
%Furthermore, we enhance the true detection rate by introducing a more strategic approach.
%%
%Experiments conducted on both synthetic and real-world datasets robustly support our theoretical results, showcasing the CAD-DA's superior performance. 
%%
%To our knowledge, this is the first work capable of conducting valid statistical inference within the context of DA.}



% \htcomment{
% say something more like "We conquer the challenge by ...." (more active tone). Also add something like "We prove that under OT-DA, a popular DA setting, conditional SI can be leveraged on its linear and quadratic operators to achieve controllable DA."

% Basically the goal is to make this part a bit more concrete than sounding like "We use an established tool to solve it in one hour"

% \red{Duy: updated}
% }
